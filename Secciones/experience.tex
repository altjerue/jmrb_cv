\section{Experiencia profesional}

%-----------------------------------
\cventry{}%
{\DPA}%
{Investigador Posdoctoral}%
{Octubre 2018 -- 2020}%
{\Purdue, EUA}%
{\textsc{Mentor}: Prof. Dimitrios Giannios%
  \begin{itemize}
    \item Creador y principal desarrollador del código de software numérico \texttt{Paramo}
    \begin{itemize}
      \item Solución numérica de la ecuación de Fokker-Planck
      \item Procesos de radiación no térmica (sincrotrón y Compton inversa) calculados numéricamente 
      \item Enfriamiento radiativo en el régimen Klein-Nishina calculado numéricamente
    \end{itemize}
    \item Orientación de estudiantes de doctorado
    \item Espectro de radiación Compton externa y evolución en el contexto de emisión tardía de brotes de rayos $\gamma$ \cite{Zhang:2020ch}
    \item Conexión entre la carga de bariones y la \emph{secuencia blazar} \cite{RuedaBecerril:2020ha}.
    \item Turbulencia y procesos de aceleración de partículas en blazares
    \item Simulación de procesos de acreción en torno a agujeros negros aislados que habitan en nuestra galaxia, usando el software numérico \texttt{HARM}
    \item Enfriamiento radiativo en flujos relativistas
  \end{itemize}
}
%-----------------------------------
\cventry{}%
{\IFMes}%
{Investigador Posdoctoral}%
{\hspace{-10ex}Enero -- Septiembre 2018}%
{\UMSNHes, México}%
{\textsc{Mentor}: Prof. Francisco S. Guzmán
  \begin{itemize}
    \item Entrenamiento de estudiantes de maestría para el uso de herramientas de software sofisticadas como el formato de almacenamiento \texttt{HDF5} \url{https://github.com/altjerue/howto_HDF5}
    \item Orientación de estudiantes de maestría
    \item Desarrollo de una herramienta en Python para el tratamiento de una gran lista de imágenes y su conversión a números para su aplicación en análisis de aprendizaje máquina (Machine Learning)
    \item Desarrollo de una herramienta en Python para la visualización de espectros, curvas de luz, etc., para el análisis de evolución espectral \url{https://github.com/altjerue/SAPyto}
  \end{itemize}
}
%-----------------------------------
\cventry{}%
{\DAAval}%
{Asistente de investigación graduado}%
{Octubre 2011 -- Julio 2017}%
{\UVval, España}%
{\textsc{Supervisores}: Prof. Miguel A. Aloy \& Dr. Petar Mimica
  \begin{itemize}
    \item Análisis de grandes bases de datos para la identificación de patrones en los espectros de emisión de blazares en el contexto del modelo de choques internos
    \item Comparación y contraste de datos obtenidos con simulaciones con la base de datos generada por el telescopio de la NASA \emph{Fermi}-LAT \cite{RuedaBecerril:2014mi}
    \item Desarrollador de software numérico con técnicas novedosas para calcular emisión ciclotrón y sincrotrón \cite{RuedaBecerril:2017mi}
  \end{itemize}
}
%-----------------------------------
\cventry{}%
{\IFMes}%
{Asistente de investigación graduado}%
{\hspace{-30ex}Agosto 2009 -- Septiembre 2011}%
{\UMSNHes, México}%
{\textsc{Supervisor}: Prof. José A. Cervera
  \begin{itemize}
    \item Desarrollador de un código SPH para la evolución de sistemas hidrodinámicos con condiciones iniciales tipo TOV
  \end{itemize}
}
%-----------------------------------
\cventry{}%
{\FCes}%
{Asistente de investigación no graduado}%
{Septiembre 2008 -- Mayo 2009}%
{\UAEMes, Mexico}%
{\textsc{Supervisor}: Prof. Francisco S. Guzmán
  \begin{itemize}
    \item Desarrollador de software numérico para resolver la ecuación de las geodésicas en métricas analíticas y numéricas \cite{Guzman:2009ru}
    \item Obtención del premio \emph{Lic. Juan Josafat Pichardo Cruz}
  \end{itemize}
}
%-----------------------------------
