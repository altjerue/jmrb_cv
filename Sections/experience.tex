%\section{Professional Experience}
\section{Experience}
%-----------------------------------
\cventry{}%
{\PYX}%
{Software Engineer (Remote)}%
{Apr 2022 -- Jan 2024}%
{Rochester NY, USA; based in Seattle, WA, USA}%
{%
\begin{itemize}
    \item Conducted data preparation, validation, and analysis in SQL from Oracle EBS datasets.
    \item Developed Java Kafka consumers for the streamlined transfer of large volumes of client data across databases. Developed high quality code using Java, Spring Boot, Kafka, PL/SQL, and deployed into production using Jenkins and OpenShift, following software development best practices.
    \item Collaborated with stakeholders, other software developers and engineers, and senior leadership to assess product needs and meet code standards for continuous integration model.
    \item Created Splunk dashboards and alerts for analysis of production data.
    \item Developed and deployed Python tests to ensure software quality and continuous integration.
\end{itemize}
}
%-----------------------------------
\cventry{}%
{\RIT}%
{Postdoctoral Research Associate}%
{Feb 2021 -- Apr 2022}%
{Rochester, NY, USA}%
{%
\begin{itemize}
    \item Led a team of specialists on a NSF-sponsored project to upgrade the C code \texttt{PatchworkMHD} to perform HPC simulations using state-of-the-art numerical techniques to model supermassive black hole binaries.
    \item Implemented a new feature (black hole spin) to \texttt{PatchworkMHD}, making more realistic binary black hole simulations without impacting runtime.
    \item Designed the experiments and evaluated state-of-the-art mathematic and numerical algorithms implemented in \texttt{PatchworkMHD} by running simulations at \emph{Frontera} supercomputer (TACC, UT at Austin).
    \item Worked in a detail-oriented manner to successfully benchmark and identify performance optimization opportunities of the scientific code.
    \item Mentored and collaborated with a graduate student to apply the machine learning algorithm \emph{gradient descent} to adjust the parameters of the open-source code, \texttt{Paramo}, to classify observations of blazars (extra-galactic objects) from Fermi-LAT telescope.
    \item Participated in a multi-institutional collaboration to study binary Neutron Star mergers through HPC simulations, resulting in 2 publications that provided critical breakthrough insights of the physics underlying these events.
    \item Published 3 co-authored papers and mentored graduate students (2 Ph.D.).
\end{itemize}
}
%-----------------------------------
%\newpage
\cventry{}%
{\Purdue}%
{Postdoctoral Research Fellow}%
{Oct 2018 -- Nov 2020}%
{West Lafayette, IN, USA}%
{%
\begin{itemize}
    \item Developed the open-source code, \texttt{Paramo}, a numerical code in Fortran 95 optimized with OpenMP to perform radiative transfer simulations in relativistic astrophysics scenarios.
    \item Obtained and led a NASA grant to explain the origin and nature of radiation from active galaxies (blazars) using  numerical and statistical models for objects observed with NASA Fermi-LAT space telescope. This research helped to unify our understanding of the two main types of blazars, identifying that important physical constraints applied to both objects.
    \item Developed Python tools to calculate the loss of energy due of high-energy particles due to interactions spectrum and evolution in the context of gamma-ray burst afterglows by developing sophisticated numerical integration, and OpenMP optimized features to \texttt{Paramo}.
    \item Collaborated with a group of multidisciplinary scientists to develop Python scripts for statistically modeling the COVID-19 outbreak in Mexico and helped create scientific infographics and blogposts for Spanish-speaking populations to reduce the spread of misinformation.
    \item Published 1 first-author and 1 co-authored paper and mentored three graduate students (1 M.S. and 2 Ph.D.).
\end{itemize}
}
%-----------------------------------
\cventry{}%
{\UMSNHes}%
{Postdoctoral Research Fellow}%
{Jan -- Sep 2018}%
{Morelia, Michoacan, Mexico}%
{%
\begin{itemize}
    \item Developed a Python script that would process images of spinning black holes simulations from  the numerical code \texttt{GRTrans} to provide an SVM with training data that would later predict radio images of actual black holes.
    \item Developed an open-source data analysis and visualization tool in Python to provide any user with accessible tools to calculate radiative transfer phenomena (spectra and light-curves) in relativistic astrophysics.
    \item Organized a workshop to train graduate students in the use of the high-volume data storage tool HDF5.
\end{itemize}
}
%-----------------------------------
\cventry{}%
{\UVval}%
{Graduate Research Assistant}%
{Oct 2011 -- Jul 2017}%
{Burjassot, Valencia, Spain}%
{%
\begin{itemize}
    \item Independently developed Shell and Python scripts to build pipelines to run simulation of the radiative transfer code \texttt{C-SPEV} and perform data processing of output datasets in HDFS format, ensuring data quality and integrity for downstream analysis and model fit with observations.
    \item Developed Python scripts to perform exploratory data analysis on datasets from NASA Fermi-LAT telescope and from the Very Large Baseline Array (VLBA) of the National Radio Astronomy Observatory (NRAO), and build non-linear regression models.
    \item Independently constructed models from \texttt{C-SPEV} simulations for curve fitting, pattern recognition, and prediction of data from NASA telescopes.
    \item Conducted multiple analyses to identify patterns in spectra and light-curves that allowed the quantification of magnetization of plasma in blazars.
    \item Conducted an analysis that identified the importance of including both cyclotron and synchrotron radiation from non-relativistic to ultra-relativistic charged particles in blazar simulations.
    \item Implemented sophisticated numerical tools and data handling to \texttt{C-SPEV} that could calculate both discrete and continuous spectra from particle distributions with arbitrary shape, without impacting simulation runtime.
    \item Published 2 first-author papers.
\end{itemize}
}
%-----------------------------------
%\cventry{}%
%{\IFMes}%
%{Research assistant}%
%{\hspace{-30ex}Aug 2009 -- Sep 2011}%
%{\UMSNHes, Michoacan, Mexico}%
%{%
%  \begin{itemize}
%    \item Developed a SPH code to evolve a hydrodynamical system in curved space-times, with TOV initial conditions
%  \end{itemize}
%}
%-----------------------------------
%\cventry{}%
%{\FCes}%
%{Research assistant}%
%{Sep 2008 -- May 2009}%
%{\UAEMes, Mexico, Mexico}%
%{%
%  \begin{itemize}
%    \item Developed a numerical solver of coupled ODEs
%    \item Priced with the \emph{Lic. Juan Josafat Pichardo Cruz} award
%  \end{itemize}
%}
%-----------------------------------
