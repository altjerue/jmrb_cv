\section{Professional Experience}

%-----------------------------------
\cventry{October 2018 -- Present}%
{\DPA}{Postdoctoral Fellow}%
{}%
{}%
{\Purdue, West Lafayette, IN, USA\\%
  \textsc{Mentor}: Prof. Dimitrios Giannios
  \begin{itemize}
    \item Creator and developer of the code \texttt{Paramo}
    \begin{itemize}
      \item Numerical Fokker-Planck equation solver
      \item Numerical non-thermal radiation processes: synchrotron and inverse Compton
      \item Numerical Klein-Nishina radiative cooling
    \end{itemize}
    \item Mentoring graduate students
    \item External Compton spectrum and evolution in the context of $\gamma$-ray burst afterglows \cite{Zhang:2020ch}
    \item Connection between the baryon loading and the so-called \emph{blazar sequence} \cite{RuedaBecerril:2020ha}.
    \item Turbulence as acceleration process in blazars using \texttt{Paramo} (work in progress)
    \item Simulations of accretion around isolated black holes using \texttt{HARM} (work in progress)
    \item Radiative cooling in relativistic outflows using \texttt{Paramo} (work in progress)
  \end{itemize}
}
%-----------------------------------
\cventry{January -- October 2018}%
{\IFMes}{Postdoctoral Fellow}%
{}%
{}%
{\UMSNHes, Morelia, Mexico\\%
  \textsc{Mentor}: Prof. Francisco S. Guzmán
  \begin{itemize}
    \item Training graduate students on computational tools, e.g., HDF5. \url{https://github.com/altjerue/howto_HDF5}
    \item Mentoring graduate students.
    \item Treatment of large number of output images from the numerical code \texttt{GRTRANS} for Machine Learning analysis.
    \item Developed the visualization tool \texttt{SAPytho} for spectral evolution. \url{https://github.com/altjerue/SAPyto}
  \end{itemize}
}
%-----------------------------------
\cventry{October 2011 -- September 2017}%
{\DAAval}{Graduate research assistant}%
{}%
{}%
{\UVval, Burjassot, Spain\\%
  \textsc{Supervisors}: Prof. Miguel A. Aloy \& Dr. Petar Mimica
  \begin{itemize}
    \item Applied the \emph{internal-shocks} model in the context of blazar flares to identify the signature of the magnetization in their SEDs. Our results were contrasted with data from the \emph{Fermi} LAT Second AGN Catalog database \cite{RuedaBecerril:2014mi}.
    \item Developed numerical technique to calculate (cyclo-)synchrotron emission from non-, trans-, and ultra-relativistic charged particles. Calculations applied to the \emph{internal-shocks} model of blazar flares \cite{RuedaBecerril:2017mi}.
  \end{itemize}
}
%-----------------------------------
\cventry{August 2009 -- September 2011}%
{\IFMes}{Graduate research assistant}%
{}%
{}%
{\UMSNHes, Morelia, Mexico\\%
  \textsc{Supervisor}: Prof. José A. Cervera
  \begin{itemize}
    \item Developer of a SPH code to evolve a hydrodynamical system with TOV initial conditions.
  \end{itemize}
}
%-----------------------------------
\cventry{September 2008 -- May 2009}%
{\FCes}{Undergraduate research assistant}%
{}%
{}%
{\UAEMes, Toluca, Mexico\\%
  \textsc{Supervisor}: Prof. Francisco S. Guzmán
  \begin{itemize}
    \item Developer of numerical null geodesic equation solver for analytical and numerical metrics \cite{Guzman:2009ru}
  \end{itemize}
}
%-----------------------------------
