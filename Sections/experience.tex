\section{Research experience}

% \cventry{2017--Present}{Postdoctoral Research Associate}{\IFMes,
% \UMSNHes}{Morelia, Mexico}{}{}

\cventry{2011--2017}{Graduate research assistant}{\UVval}{Burjassot, Spain}{}{
\begin{itemize}
  \item Studied in depth AGNs and blazars, the blazars models, the radiation
  transfer equation, the kinetic equation.
  \item Automatized the launching of simulations, treatment of data and
  generation of plots for an extensive parameter space study of the internal
  shocks code developed by Petar Mimica and Miguel A. Aloy in order to find
  traces left in the spectra due to the magnetization of the shocked shells of
  plasma.
  \item Extracted and interpreted data from the simulations of the main
  characteristics of blazars SEDs, e.g. Compton dominance, syncrotron and
  Compton peaks, spectral index.
  \item Extracted, cleaned and processed data from the \emph{Fermi} LAT Second
  AGN Catalog database for the comparison with our simulations.
  \item I implemented a routine for a more general distribution of particles
  (thermal and nonthermal) injected at the shock front to be treated in the
  original code.
  \item I calculated tables with the \mbs emission of charged particles of
  arbitrary velocity, and the emissivity for isotropic distributions of
  electrons using a code that I developed from scratch.
  \item I implemented the \mbs tables to the original code and performed
  simulations of the internal shocks scenario for blazars.
  \item Interpreted the new SEDs out of the simulations.
  \item Wrote and defended a thesis.
  \item Contributed to the writing of and coauthored two manuscript for publication in a peer-reviewed journal.
\end{itemize}
}
%
\cventry{2010--2011}{Graduate research assistant}{\IFMes}{Morelia, Mexico}{}{%
\begin{itemize}
  \item I developed a newtonian and relativistic smoothed-particle hydrodynamics
  (SPH) codes.
  \item I solved the TOV field equations numerically using my RK4 solver.
  \item I wrote a routine with the simple predictor-corrector method: Euler
  method with the trapezoidal rule.
  \item I simulated a TOV star using the numerical solution of the TOV field
  equations as initial conditions of the SPH code and evolved the system using
  the predictor-corrector routine.
  \item Wrote and presented a master thesis with the results obtained.
\end{itemize}
}
%
\cventry{2008--2009}{Graduate research assistant}{\UAEMes}{Toluca, Mexico}{}{%
 \begin{itemize}
  \item Wrote and characterized a fourth-order Runge-Kutta (RK4) solver for
  analytic and numeric input functions for each stage.
  \item I solved the null geodesic equation for two spherically symmetric and
  static space-times using the RK4 solver: black holes (analytic Christoffel
  symbols) and boson stars (numeric Christoffel symbols).
  \item I simulated and interpreted light trajectories due to curved space-times
  and characterized such trajectories for gravitational lenses.
  \item Wrote and presented a thesis with the results obtained.
  \item Contributed to the writing of and coauthored a manuscript for
  publication in a peer reviewed journal.
\end{itemize}
}

%
\cventry{2007--2008}{Undergraduate research assistant}{\UAEMes}{Toluca, Mexico}{}{Internship service project, supervised by Prof. Jorge Orozco Velasco.%
\begin{itemize}
  \item Writing the elliptic equations in finite differences form
  \item Characterization of the typical kinds of boundary conditions:
  \begin{itemize}
    \item Dirichlet
    \item Neumann
  \end{itemize}
  \item Writing of a code which solves the two-dimensional Laplace equation in Cartesian coordinates with Dirichlet and Neumann boundary conditions.
\end{itemize}
}
%
\cventry{25 Jun--24 Aug 2007}{Undergraduate research assistant}{Mexican Academia of Science}{Morelia, Mexico}{}{National program for temporary stays at national research centers for undergraduate science students.\newline%
Supervisor: Prof. Francisco S. Guzm\'{a}n Murillo.
\begin{itemize}
  \item Numerical solution of the wave equation with finite differences.
  \item Numerical solution of Burgers' equation with finite differences.
  \item Numerical solution of the general relativistic one-dimensional wave equation in the 3+1 formalism with finite differences.
\end{itemize}
}
%
\cventry{2005--2008}{Undergraduate researcher assistant}{\UAEMes}{Toluca, Mexico}{}{Volunteer work in a faculty research project\newline%
Supervisor: Prof. Porfirio D. Rosendo-Francisco
\begin{itemize}
  \item Exposure of graphite samples to microwaves
  \begin{itemize}
    \item Ultrasonic cleaning of graphite samples.
    \item Systematic exposure graphite samples to microwaves ($\SI{2.45}{\giga\hertz}$).
    \item Observation of the superficial effects using a metallographic microscope.
    \item Characterization of the structures observed.
    % \item Results presented in a poster at the XLVIII National Physics
    %   Meeting, Guadalajara, M\'{e}xico, 2005.
  \end{itemize}
  \item Exposure of graphite samples to elecric arcs%
  \begin{itemize}
    \item Ultrasonic cleaning of graphite samples.
    \item Characterization of a Tesla coil.
    \begin{itemize}
      \item Input current.
      \item Output flux of electrons.
    \end{itemize}
    \item Controlled handling of a Tesla coil.
    \item Systematic exposure of the surface of graphite samples to a perpendicular and tangential electric arc.
    \item Observation of surface effects with a metallographic microscope.
    \item Characterization of the zones around the contact region.
    \item Characterization of the temperature around the contact region.
    \item Characterization of the structures which appeared after the exposure.
    \item Analysis of X-rays spectra of the samples.
    \item Identification of induced families of lattice planes.
    % \item Results presented in a poster at the XLIX National Physics Meeting, San Luis Potosi, M\'{e}xico, 2006.
    % \item Results presented in a poster at the L National Physics Meeting, Boca del R\'{\i}o, M\'{e}xico, 2007.
  \end{itemize}
\end{itemize}
}
%
% \cventry{year--year}{Job title}{Employer}{City}{}{Description line 1\newline{}Description line 2
% \item Achievement 2, with sub-achievements:
%   \begin{itemize}%
%   \item Sub-achievement (a);
%   \item Sub-achievement (b), with sub-sub-achievements (don't do this!);
%     \begin{itemize}
%     \item Sub-sub-achievement i;
%     \item Sub-sub-achievement ii;
%     \item Sub-sub-achievement iii;
%     \end{itemize}
%   \item Sub-achievement (c);
%   \end{itemize}
% \item Achievement 3.
% }
%
% \subsection{Miscellaneous}
% \cventry{year--year}{Job title}{Employer}{City}{}{Description}
