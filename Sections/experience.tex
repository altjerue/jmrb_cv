\section{Professional Experience}

%-----------------------------------
\cventry{}%
{Paychex (on contract through Mindex)}%
{Software Engineer (Remote)}%
{Apr 2022 -- Present}%
{Rochester NY, USA; based in Seattle, WA, USA}%
{%
\begin{itemize}
	\item Contributed on building the essential blocks of the infrastructure of QTC+ Process by performing data analysis and develop sophisticated PL/SQL scripts.
	\item Collaboration on developing, and deploying Spring boot services in Java to transfer large volumes of data with Kafka.
	\item Expertise on using Jenkins automation server, Docker and Podman containers, and OpenShift orchestration tool.
	\item Expertise on continuous integration by being involved at the different stages of the process: developing micro-services, developing building tools, and developing automated test tools.
	\item Continuously communicating with stakeholders to ensure developers efforts are aligned with requirements and standards.
	\item Member of a Scrum team, following the Agile methodology using Jira.
	\item Mentoring junior developers and helped grow the team.
\end{itemize}
}
%-----------------------------------
\cventry{}%
{\CCRG}%
{Postdoctoral Research Associate}%
{Feb 2021 -- Present}%
{\RIT, NY, USA}%
{%
  \begin{itemize}
    \item Operate the HPC code \texttt{PatchworkMHD}, written in C, to model supermassive black holes
    \item Debugged, benchmarked and tested scalability of \texttt{PatchworkMHD}
    \item Version control administrator of \texttt{PatchworkMHD}
  \end{itemize}
}

%-----------------------------------
\cventry{}%
{\DPA}%
{Postdoctoral Fellow}%
{October 2018 -- November 2020}%
{\Purdue, USA}%
{\textsc{Mentor}: Prof. Dimitrios Giannios%
  \begin{itemize}
    \item Designed and developed the code \texttt{Paramo}, a numerical code in Fortran, optimized with OpenMP parallelization
    \item Led a project, funded by NASA, to model the origin and nature of the radiation from active galaxies \cite{RuedaBecerril:2020ha}, finding a connection between the baryon loading and the so-called \emph{blazar sequence}
    \item Contrasted simulations with data from the space telescope \emph{Fermi}-LAT \cite{RuedaBecerril:2021ha}
    \item Mentored graduate students
    \item External Compton spectrum and evolution in the context of $\gamma$-ray burst afterglows \cite{Zhang:2020ch}
    \item Volunteered on the development of simple models of the COVID-19 breakout, at the beginning of the pandemic, along with scientific infographics and blogposts for Spanish speaking countries, particularly Mexico, to stop the spread of misinformation
  \end{itemize}
}
%-----------------------------------
\cventry{}%
{\IFMes}%
{Postdoctoral Fellow}%
{\hspace{-30ex}January -- September 2018}%
{\UMSNHes, Mexico}%
{\textsc{Mentor}: Prof. Francisco S. Guzmán
  \begin{itemize}
    \item Trained graduate students on computational tools, e.g., HDF5. \url{https://github.com/altjerue/howto_HDF5}
    \item Mentored graduate students.
    \item Developed a Python script to treat large number of output images from the numerical code \texttt{GRTRANS} for Machine Learning analysis.
    \item Developed the visualization tool \texttt{SAPytho} for spectral evolution. \url{https://github.com/altjerue/SAPyto}
  \end{itemize}
}
%-----------------------------------
\cventry{}%
{\DAAval}%
{Graduate research assistant}%
{October 2011 -- July 2017}%
{\UVval, Spain}%
{\textsc{Supervisors}: Prof. Miguel A. Aloy \& Dr. Petar Mimica
  \begin{itemize}
    \item Learned and operated the scientific numerical code \texttt{C-SPEV}, written in Fortran, to apply the \emph{internal-shocks} model to blazar flares
    \item Developed Python tools for data fitting, and identifying patterns by manipulating and analyzing large amounts of data sets
    \item Employed data from the space telescope \emph{Fermi}-LAT for fitting and contrasting the parameter space of our model
    \item Applied the \emph{internal-shocks} model to blazar flares
    \item Identified spectral signatures of magnetization.
    \item Contrasted simulations with data from the \emph{Fermi}-LAT Second AGN Catalog database \cite{RuedaBecerril:2014mi}.
    \item Developed analysis tools in Julia
    \item Developed numerical tools to \texttt{C-SPEV} to calculate (cyclo-)synchrotron emission
    \item Developed Bash script and Python tools to automatize the manipulation and analysis of large data sets
    \item Version control administrator of \texttt{C-SPEV}
    \item Calculated (cyclo-)synchrotron emission from non-, trans-, and ultra-relativistic charged particles \cite{RuedaBecerril:2017mi}.
  \end{itemize}
}
%-----------------------------------
\cventry{}%
{\IFMes}%
{Graduate research assistant}%
{\hspace{-30ex}August 2009 -- September 2011}%
{\UMSNHes, Mexico}%
{\textsc{Supervisor}: Prof. José A. Cervera
  \begin{itemize}
    \item Developer of a numerical code, based on the SPH method, to evolve a hydrodynamical system with TOV initial conditions.
  \end{itemize}
}
%-----------------------------------
\cventry{}%
{\FCes}%
{Undergraduate research assistant}%
{September 2008 -- May 2009}%
{\UAEMes, Mexico}%
{\textsc{Supervisor}: Prof. Francisco S. Guzmán
  \begin{itemize}
    \item Developer of a numerical code to solve the null geodesic equation in analytical and numerical metrics \cite{Guzman:2009ru}
    \item Priced with the \emph{Lic. Juan Josafat Pichardo Cruz} award.
  \end{itemize}
}
%-----------------------------------
