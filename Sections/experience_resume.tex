\section{Experience}
%--------------------
\cventry{}%
{TealWaters}%
{Spatial Data Scientist}%
{May 2025 — Present}%
{}%
{%
\begin{itemize}
  \item Lead project to optimize code for topographic terrain modeling and elevation derivative calculation.
  \item Engineered Python tools for geospatial data processing and analysis, utilizing QGIS for visualization and analysis.
  \item Utilize Wetland Intrinsic Potential (WIP) tool with random forest for wetland probability mapping in Skykomish watershed.
  \item Performed EDA, feature engineering, and geospatial data preparation—including multispectral (Sentinel 1/2) processing—for ML/AI land-cover and wetland probability models.
  \item Share knowledge and results with managers and decision-makers, train team members on WIP tool usage, and collaborate with software engineers and scientific team to transition prototypes into production.
\end{itemize}
}
%--------------------
\cventry{}%
{\vspace{-10pt}}%
{Independent Research/Open-Source Developer}%
{Jan 2024 — May 2025}%
{}%
{%
\begin{itemize}
  \item Developed scientific codes (Tleco, WindsOfChange), completed ML/AI coursework, and contributed to open-source geospatial and scientific-computing tools.
\end{itemize}
}
%--------------------
\cventry{}%
{\PYX}%
{Software Engineer}%
{Apr 2022 — Jan 2024}%
{}%
{%
\begin{itemize}
  \item Conduct data preparation, validation, and analysis in SQL from Oracle EBS datasets.
  \item Built Java Kafka consumers for efficient data transfer across databases; deploy using Jenkins and OpenShift.
  \item Collaborate with stakeholders and engineers to meet product needs and code standards.
  \item Create Splunk dashboards and alerts for production data analysis.
%  \item Develop and deploy Python tests for software quality and continuous integration.
\end{itemize}
}
%--------------------
\cventry{}%
{\UMSNH\ (Mexico), \Purdue, \RIT}%
{Postdoctoral Research Scientist}%
{Jan 2018 — Apr 2022}%
{}%
{%
\begin{itemize}
  \item Lead NSF-sponsored project to upgrade scientific code for HPC simulations of black hole binaries; benchmark and optimize performance.
  \item Collaborate on large-scale HPC simulations of neutron star mergers; produce 2 high-impact papers.
  \item Obtain \$68,000 NASA grant as primary researcher; manage 3 Ph.D. researchers; published 3 papers, 2 proceedings, and 5 presentations.
  \item Designed large-scale computational experiments and analytical workflows for astrophysical data and simulation outputs.
%  \item Participate in Purdue Astronomy Journal Club discussions.
  \item Designed Python script for machine learning (SVM) training framework.
  \item Conduct workshop on HDF5 dataset creation and manipulation for graduate students.
\end{itemize}
}
%--------------------
\cventry{}%
{\UMSNH\ (Mexico), \UVval\ (Spain)}%
{Graduate Research Assistant}%
{Oct 2009 — Jul 2017}%
{}%
{%
\begin{itemize}
  \item Automated Shell and Python tools for data processing and curation in HDF5 format; ensure data quality and integrity.
  \item Implement numerical tools for calculating spectra from particle distribution functions.
  \item Use Python and R for exploratory data analysis on NASA telescope datasets; yield 2 papers and present at international meetings.
  \item Enhanced scientific code using Runge-Kutta solver for light behavior near black holes; yield high-impact paper.
%  \item Attend workshops on Data Analysis and Machine Learning with Python.
\end{itemize}
}
