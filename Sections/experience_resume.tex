\section{Experience}
%-----------------------------------
\cventry{}%
{\PYX}%
{Software Engineer}%
{Apr 2022 -- Jan 2024}%
{}%
{%
\begin{itemize}
    \item Developed high quality code using Java, Bootstrap, Kafka, PL/SQL, and deployed into production using Jenkins and OpenShift, following software development best practices.
    \item Collaborated with stakeholders and other software development teams to assess product needs and meet code standards for continuous integration model.
    \item Developed Java Kafka consumers for the streamlined transfer of large volumes of client data across databases.
    \item Conducted data processing and extensive quantitative analysis in PL/SQL and created Splunk dashboards and alerts for analysis of production data.
    \item Developed and deployed Python tests to ensure software quality and continuous integration.
\end{itemize}
}
%-----------------------------------
\cventry{}%
{\UMSNH\ (Mexico), \Purdue, \RIT}%
{Research Scientist}%
{Jan 2018 -- Apr 2022}%
{}%
{%
\begin{itemize}
    \item Led a team of specialists on a NSF-sponsored project and successfully upgraded a hydrodynamic code to perform HPC simulations of supermassive black hole binaries. Designed the experiments and evaluated state-of-the-art mathematic and numerical techniques implemented in the code by running simulations at \emph{Frontera} supercomputer (TACC, UT at Austin). Worked in a detail-oriented manner to successfully benchmark and identify performance optimization opportunities of the code.
    \item Worked in a highly collaborative environment with multi-institutional, cross-functional teams developing large-scale HPC simulations of neutron star mergers. This collaboration produced 2 papers with high impact results.
    \item Successfully applied and obtained a \$68,000 NASA grant for one year as primary researcher. Managed 3 Ph.D. researchers to study the nature of radiation from active galaxies with state-of-the-art mathematical and numerical methods. This work produced 3 papers and 2 proceedings with high impact results, and 5 presentations to technical audiences.
    \item Participated in weekly paper discussion and knowledge sharing at the Purdue Astronomy Journal Club.
    \item Developed a Python script that would process images for a machine learning (SVM) training framework. Conducted a workshop to train and share knowledge with graduate students on creation and manipulation of high-volume datasets in HDF5 format.
\end{itemize}
}
%-----------------------------------
\cventry{}%
{\UMSNH\ (Mexico), \UVval\ (Spain)}%
{Graduate Research Assistant}%
{Oct 2009 -- Jul 2017}%
{}%
{%
\begin{itemize}
    \item Developed Shell and Python tools for automation and pipelines for data processing and curation of large volume datasets in HDF5 format, ensuring data quality and integrity for downstream analysis.
    \item Developed Python scripts to perform exploratory data analysis on datasets from NASA telescopes. Conducted diverse analyses to identify patterns in spectra and light-curves that allowed to make insightful predictions. This work produced 2 papers and communicated my results to technical audiences.
    \item Developed a numerical code to study the behavior of light near black holes. This work produced one paper with high impact results.
\end{itemize}
}
