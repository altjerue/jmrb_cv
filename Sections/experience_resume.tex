\section{Experience}
%-----------------------------------
\cventry{}%
{\PYX}%
{Software Engineer}%
{Apr 2022 -- Jan 2024}%
{}%
{%
\begin{itemize}
    \item Conducted data preparation, validation, and analysis in SQL from Oracle EBS datasets.
    \item Developed Java Kafka consumers for the streamlined transfer of large volumes of client data across databases. Developed high quality code using Java, Spring Boot, Kafka, PL/SQL, and deployed into production using Jenkins and OpenShift, following software development best practices.
    \item Collaborated with stakeholders, other software developers and engineers, and senior leadership to assess product needs and meet code standards for continuous integration model.
    \item Created Splunk dashboards and alerts for analysis of production data.
    \item Developed and deployed Python tests to ensure software quality and continuous integration.
\end{itemize}
}
%-----------------------------------
\cventry{}%
{\UMSNH\ (Mexico), \Purdue, \RIT}%
{Postdoctoral Research Scientist}%
{Jan 2018 -- Apr 2022}%
{}%
{%
\begin{itemize}
    \item Led a team of specialists on a NSF-sponsored project and successfully upgraded a scientific code to perform HPC simulations of supermassive black hole binaries. Designed the experiments and evaluated state-of-the-art mathematic and numerical algorithms implemented in a scientific code by running simulations at \emph{Frontera} supercomputer (TACC, UT at Austin). Worked in a detail-oriented manner to successfully benchmark and identify performance optimization opportunities of the scientific code.
    \item Worked in a highly collaborative environment with multi-institutional, cross-functional teams developing large-scale HPC simulations of neutron star mergers. This collaboration produced 2 papers with high impact results.
    \item Successfully applied and obtained a \$68,000 NASA grant for one year as primary researcher. Managed 3 Ph.D. researchers to study the nature of radiation from active galaxies with state-of-the-art mathematical and numerical methods. This work produced 3 papers and 2 proceedings with high impact results, and 5 presentations to technical audiences.
    \item Designed the experiments to apply mathematical and numerical methods to prove a hypothesis about the origin and nature of radiation from active galaxies. Used Python to run regressions on observations from NASA telescopes.
    \item Participated in weekly paper discussion and knowledge sharing at the Purdue Astronomy Journal Club.
    \item Developed a Python script that processed images for a machine learning (SVM) training framework.
    \item Conducted a workshop to train and share knowledge with graduate students on creating and manipulating high-volume datasets in HDF5 format.
\end{itemize}
}
%-----------------------------------
\cventry{}%
{\UMSNH\ (Mexico), \UVval\ (Spain)}%
{Graduate Research Assistant}%
{Oct 2009 -- Jul 2017}%
{}%
{%
\begin{itemize}
    \item Developed Shell and Python tools for automation and pipelines for data processing and curation of large volume datasets in HDF5 format from large-scale simulations, ensuring data quality and integrity for downstream analysis.
    \item Implemented sophisticated numerical tools and data handling to calculate both discrete and continuous spectra from particle distribution functions with arbitrary shape, without impacting simulation runtime.
    \item Used Python and R for exploratory data analysis on datasets from NASA telescopes, performing linear and non-linear regression, pattern recognition, and forecasting. This work produced 2 papers and communicated my results to technical audiences in multiple international meetings.
    \item Successfully developed a scientific code using a fourth order Runge-Kutta solver to study the behavior of light near black holes. This work produced one scientific paper with high impact results.
    \item Attended workshops on Data Analysis and Machine Learning with Python.
\end{itemize}
}
