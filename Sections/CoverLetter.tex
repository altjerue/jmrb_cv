% {\fancyfoot[r]{}            % No page/lastapage in the letter
% \setcounter{page}{0}        % Change if letter is more than 1 page

%%%   Recipient Data
\recipient{Extreme Astrophysics Group}{Steward Observatory\\University of Arizona\\Tucson, Arizona, USA}
\date{\today}
\opening{To whom it may concern,}
\closing{Best regards,}
\enclosure[Attached]{curriculum vitae \& research statement} % use an optional argument to use a string other than "Enclosure", or redefine \enclname
%
\makeletterhead%

I recently found the announcement at the AAS job listing of a Postdoctoral
Research Fellow at the University of Arizona and I would like to apply for that
position.

I have recently finished my PhD at the University of Valencia (SPAIN), under the
supervision of Prof. Miguel \'{A}ngel Aloy and Dr. Petar Mimica. My PhD research
focused on the nature of blazars, particularly on connecting their measured
spectral features and lightcurves to the physics of the underlying plasma. I
worked within the \emph{internal shocks} paradigm. According to this model,
shocks inside a heterogeneous beam of relativistic plasma accelerate charged
particles, such as leptons and/or hadrons, to higher energies. Since the plasma
is threaded by both randomly-oriented small-scale and ordered large-scale
magnetic fields, \mbs (MBS) emission is naturally produced. For simplicty, this
radiation is considered as synchrotron for high frequencies and relativistic
electrons. The radiation formalism I appied in my thesis is more general and
deals with the cyclotron emission as well. My work consisted initially in
performing and analyzing numerical simulations based on the internal shocks
model, followed by an improvement and extension of the existing numerical tools
available in my research group at the \UVval. The spectral shape of blazars at
high frequencies in the working model is determined by inverse-Compton
upscattered photons from an external radiation field or those produced by MBS.
Due to the non-linear and complex character of the plasma dynamics, governed by
the equations of relativistic magneto-hydrodynamics, as well as the processes of
particle acceleration and MBS emission, a fully numerical modeling was
accomplished. I performed a systematic parameter coverage of the physical
properties of the radiation emitting plasma, with special emphasis on the plasma
magnetization. The kind of numerical simulations I performed during my PhD
involved solving the kinetic and radiative transfer equations in a magnetized
medium. What was found with this study were the fingerprints of the magnetic
field in the underlying plasma left in the shape of the spectra detectected by a
distant observer.

I am currently a member of the Computational Physics Group at the \IFMen\ at the
\UMSNHes\ in Mexico. My purpose here is to support state of the art simulations
of relativistic radiation hydrodynamics to model long GRBs performed by members
of the group, contributing with astrophysical insight given the expertise I
gained during my doctoral studies. Simultaneously, I am following up on two
promising research projects derived from my PhD thesis. One of the projects
consist on the construction of numerical code which consistently computes the
synchrotron-self Compton cooling term out of a hybrid thermal--nonthermal
distribution  of accelerated particles, injected into a magnetized medium.

My undergraduate work on Numerical Relativity and my ensuing involvement in
Numerical Astrophysics have helped me to develop advanced programming skills and
intuition in the performance of numerical codes, as well as great interest and
knowledge about general relativity and (magneto)hydrodynamics. Nowadays I am
mostly interested in high energy physics in astrophysical scenarios,
specifically those in the vicinity of compact objects, for instance stellar and
supermassive black holes, and neutron stars. Since the first observations by the
Event Horizon Telescope, I have been closely following the results from Sgr
A$^{*}$ and M87. I am eager to investigate the physics in the surroundings of
those objects, with a major interest on the high energy processes.

I am convinced that the experience and knowledge I have will prove highly useful
and benefitial to your research program if I should be selected. I can provide
more details about any aspect of my work/resume you are interested in. I am
available for videoconference interview any weekday.

I look forward to hearing from you soon.

\makeletterclosing%

%\newpage%
% this is the page "1/1" % (no number when total pages = 1)
%}% end of pages without page/lastapage} 
