%%%   Recipient Data
\recipient{Someone}{Somewhere}
\date{\today}
\opening{Dear,}
\closing{Best regards,}
\enclosure[Attached]{curriculum vit\ae{}} % use an optional argument to use a string other than "Enclosure", or redefine \enclname

%
\makelettertitle%

I recently found the announcement at <Fill> job listing of a Postdoctoral Research Fellow at the University of Arizona and I would like to apply for that position.

I have recently finished my PhD at the University of Valencia (SPAIN), under the supervision of Prof. Miguel \'{A}ngel Aloy and Dr. Petar Mimica. The main part of my research was on the nature of blazars, where I was working on connecting their measured spectral features and lightcurves to the physics of the underlying plasma. I was working within the \emph{internal shocks} paradigm. According to this model, shocks inside of an heterogeneous beam of relativistic plasma accelerate leptons (or even hadrons in some variants of the model) to high energies. Since the plasma is threaded by both small-scale (randomly oriented) and large-scale magnetic fields (ordered, dealt with by a relativistic magnetohydrodynamics solver), \mbs (MBS) emission is naturally produced. In practice, this radiation is synchrotron for high frequencies and relativistic electrons, though my formalism is general and deals with the cyclotron emission as well. Inverse Compton upscattering of either external photon fields or the produced MBS photons shapes the high-frequency emission of blazars in the working model. Due to the non-linear and complex character of the plasma dynamics (governed by the equations of relativistic magneto-hydrodynamics) as well as the processes of particle acceleration and MBS emission, a fully numerical modeling has been performed. During my PhD, I have improved the existing numerical tools available in my host group at the University of Valencia, both extending the previous plasma radiation mechanisms and the analysis tools to perform systematic parameter coverage of the physical properties of plasma from which the emission results (with special emphasis on the plasma magnetization). The kind of numerical simulations I have performed during my PhD involved solving the kinetic and radiative transfer equations in a magnetized medium \citeRef{Rueda:2014mn,Rueda:2017hd,Rueda:2017phd}.

At the moment I am a member of the Computational Physics Group at the \IFMen{} at the \UMSNHen{} in Mexico. My purpose here is to support state of the art simulations of relativistic radiation hydrodynamics to model long GRBs performed by members of the group \citeRef{Rivera:2016gu}, contributing with astrophysical insight given the expertise gained during my years as PhD student. Parallel to that, I am following up on research that did not enter into the final draft of my PhD thesis. In this regard, one of the projects consist on the construction of numerical code which consistently computes the synchrotron-self Compton cooling term out of a hybrid thermal--nonthermal  distribution  of accelerated particles, injected into a magnetized medium.

Since I started working on Numerical Relativity in my undergraduate years until now that I am involved in Numerical Astrophysics, I have developed high programming skills and intuition with performance of numerical codes. And since I arrived to the Department of Astronomy and Astrophysics of the University of Valencia, I have been attracted and involved in astrophysics. Nowadays I am highly interested on the high energy physics in astrophysical scenarios, more specifically due to the presence and in the vicinity of compact objects: black holes (both stellar and supermassive) and Neutron stars. Since the first observations by the Event Horizon Telescope I have been following the news waiting for the results from those observations of Sgr A$^{*}$ and M87. I am eager to investigate the physics in the surrounding of those objects, with a major interest on the high energy processes. Since the first observation of gravitational waves from a nutron stars merger and its the electromagnetic counterpart I have been following the news and the works around such an interesting breakthrough. Besides, I am eager to investigate the high energy and radiation processes that were involved during the merger, what we are able to see from electromagnetic waves and which we cannot.

I am convinced that the experience and knowledge I have will prove highly useful and valuable to your research program if I should be selected. I can provide more details about any aspect of my work/resume you are interested in. I am available for videoconference interview every weekday.

I am looking forward to hearing from you soon.


\makeletterclosing%

\bibliographystyleRef{naturemag}
\bibliographyRef{MyPapers}
