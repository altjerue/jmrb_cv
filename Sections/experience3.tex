%\section{Professional Experience}
\section{\large Professional Experience}

%-----------------------------------
\cventry{}%
{Paychex (on contract through Mindex)}%
{Software Engineer (Remote)}%
{Apr 2022 -- Present}%
{Rochester NY, USA; based in Seattle, WA, USA}%
{%
\begin{itemize}
	\item Contribution on building essential blocks for the infrastructure of QTC+ Process by performing data analysis and develop sophisticated PL/SQL scripts.
	\item Collaboration on developing, and deploying Spring boot services in Java to transfer large volumes of data with Kafka.
	\item Expertise on using Jenkins automation server, Docker and Podman containers, and OpenShift orchestration tool.
	\item Expertise on continuous integration by being involved at the different stages of the process: developing micro-services, developing building tools, and developing automated test tools.
	\item Continuously communicating with stakeholders to ensure developers efforts are aligned with requirements and standards.
	\item Member of a Scrum team, following the Agile methodology using Jira.
	\item Mentoring junior developers and helped grow the team.
\end{itemize}
}
%-----------------------------------
\cventry{}%
{\RIT}%
{Postdoctoral Research Associate}%
{Feb 2021 -- Apr 2022}%
{Rochester, NY, USA}%
{%
\begin{itemize}
	\item Led a team of specialists on a NSF-sponsored project to upgrade the HPC code PatchworkMHD, using state-of-the-art numerical techniques to model supermassive black hole binaries.
	\item Added a new feature (spin) to \texttt{PatchworkMHD}, making a more realistic binary black hole simulations, without impacting runtime.
	\item Debugged, benchmarked, and tested the scalability of PatchworkMHD.
	\item Version control (git) administrator of PatchworkMHD.
	\item Worked alongside a graduate student to develop a modified gradient descent (deep learning) algorithm on top of the Paramo code to model observations from Fermi-LAT telescope with.
	\item Mentored graduate students.
\end{itemize}
}
%-----------------------------------
\cventry{}%
{\Purdue}%
{Postdoctoral Research Fellow}%
{Oct 2018 -- Nov 2020}%
{West Lafayette, IN, USA}%
{%
\begin{itemize}
	\item Authored and co-authored 3 publications by designing, and developing the open-source code \texttt{Paramo}, a numerical code in Fortran, optimized with OpenMP.
	\item Obtained a NASA grant to explain the origin and nature of radiation from active galaxies. In this project I led a small team to develop numerical and statistical models for objects observed with NASA Fermi-LAT space telescope.
	\item Mentored three graduate students (1 M.S. and 2 Ph.D.).
	\item Calculate the loss of energy due of high-energy particles due to interactions spectrum and evolution in the context of gamma-ray burst afterglows by developing sophisticated, and OpenMP optimized, numerical method.
	\item Contributed to the development of statistical models of the COVID-19 outbreak at the beginning of the pandemic along with scientific infographics and blogposts for Spanish-speaking populations to stop the spread of misinformation.
\end{itemize}
}
%-----------------------------------
\cventry{}%
{\UMSNHes}%
{Postdoctoral Research Fellow}%
{\hspace{-30ex}Jan -- Sep 2018}%
{Morelia, Michoacan, Mexico}%
{%
\begin{itemize}
	\item Generated training data for a support vector machine by developing a Python script that would process images of spinning black holes generated by the numerical code \texttt{GRTrans}.
	\item Developed an open-source data analysis and visualization tool for radiative astrophysics.
	\item Organized a workshop to train graduate students in the use of the high-volume data storage tool HDF5.
\end{itemize}
}
%-----------------------------------
\cventry{}%
{\UVval}%
{Graduate Research Assistant}%
{Oct 2011 -- Jul 2017}%
{Burjassot, Valencia, Spain}%
{%
\begin{itemize}
	\item Published 2 first-author papers, participated in international meetings and co-authored several papers/proceedings, by running and upgrading the scientific numerical code \texttt{C-SPEV} to apply the \emph{internal-shocks} (IS) model to simulate blazar flares.
	\item Analyzed simulations with data from NASA telescopes by performing exploratory data analysis on data from the telescopes.
	\item Identified spectral signatures of magnetization in blazar spectra by employing data analysis skills, statistics and machine learning (linear and non-linear regression).
	\item Developed analysis tools in Julia. Version control (git) administrator of C-SPEV.
	\item Both cyclotron and synchrotron radiation from non-relativistic to ultra-relativistic charged particles were considered in my simulations by implementing in C-SPEV a sophisticated numerical tool that could calculate both discrete and continuous spectra.
	\item Included low energy particles to the statistical system in the IS model, without impacting simulation runtime, by implemented complex numerical integration methods and fitting into C-SPEV.
\end{itemize}
}
%-----------------------------------
%\cventry{}%
%{\IFMes}%
%{Research assistant}%
%{\hspace{-30ex}Aug 2009 -- Sep 2011}%
%{\UMSNHes, Michoacan, Mexico}%
%{%
%  \begin{itemize}
%    \item Developed a SPH code to evolve a hydrodynamical system in curved space-times, with TOV initial conditions
%  \end{itemize}
%}
%-----------------------------------
%\cventry{}%
%{\FCes}%
%{Research assistant}%
%{Sep 2008 -- May 2009}%
%{\UAEMes, Mexico, Mexico}%
%{%
%  \begin{itemize}
%    \item Developed a numerical solver of coupled ODEs
%    \item Priced with the \emph{Lic. Juan Josafat Pichardo Cruz} award
%  \end{itemize}
%}
%-----------------------------------
