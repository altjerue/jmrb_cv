\documentclass[11pt,a4paper,sans]{moderncv}        % possible options include font size ('10pt', '11pt' and '12pt'), paper size ('a4paper', 'letterpaper', 'a5paper', 'legalpaper', 'executivepaper' and 'landscape') and font family ('sans' and 'roman')


% moderncv themes
\moderncvstyle{classic}                             % style options are 'casual' (default), 'classic', 'banking', 'oldstyle' and 'fancy'
\moderncvcolor{blue}                               % color options 'black', 'blue' (default), 'burgundy', 'green', 'grey', 'orange', 'purple' and 'red'
%\renewcommand{\familydefault}{\sfdefault}         % to set the default font; use '\sfdefault' for the default sans serif font, '\rmdefault' for the default roman one, or any tex font name
%\nopagenumbers{}                                  % uncomment to suppress automatic page numbering for CVs longer than one page

% character encoding
%\usepackage[utf8]{inputenc}                       % if you are not using xelatex ou lualatex, replace by the encoding you are using

\usepackage{url}
\usepackage{uri}
\usepackage[abbreviations=false]{siunitx}
\sisetup{%
  retain-unity-mantissa = false,% avoid 1 x 10^{4}
  range-units = single,% do not repeat units, e.g.: 2,3 and 4 T
  range-phrase = --% 
}

\usepackage{amssymb}
\usepackage{amsmath}

% adjust the page margins
\usepackage[scale=0.8]{geometry}
\recomputelengths{}
%\setlength{\hintscolumnwidth}{3cm}                % if you want to change the width of the column with the dates
%\setlength{\makecvtitlenamewidth}{10cm}           % for the 'classic' style, if you want to force the width allocated to your name and avoid line breaks. be careful though, the length is normally calculated to avoid any overlap with your personal info; use this at your own typographical risks...

% personal data
\name{Jes\'{u}s Misr\'{a}yim}{Rueda-Becerril}
\title{PhD candidate}                             % optional, remove / comment the line if not wanted
\address{Av. Paseo Rajolar 2, Piso 7}{46100, Burjassot}{Valencia, SPAIN}  % optional, remove / comment the line if not wanted; the "postcode city" and "country" arguments can be omitted or provided empty
\phone[mobile]{+34~61~572~8749}                   % optional, remove / comment the line if not wanted; the optional "type" of the phone can be "mobile" (default), "fixed" or "fax"
%\phone[fixed]{}
%\phone[fax]{}
\email{jesus.rueda@uv.es}
%\homepage{}                         % optional, remove / comment the line if not wanted
\social[linkedin]{Jesus Rueda-Becerril}          % optional, remove / comment the line if not wanted
%\social[twitter]{jerue103}                             % optional, remove / comment the line if not wanted
\social[github]{altjerue}                              % optional, remove / comment the line if not wanted
%\photo[64pt][0.4pt]{picture}                       % optional, remove / comment the line if not wanted; '64pt' is the height the picture must be resized to, 0.4pt is the thickness of the frame around it (put it to 0pt for no frame) and 'picture' is the name of the picture file
%\quote{}                                 % optional, remove / comment the line if not wanted

%% -------------------------------  BIBLIOGRAPHY  --------------------------

%    bibliography with mutiple entries
\usepackage{multibib}
\newcites{jour,conf}{{Articles}, {Proceedings}}
\usepackage[numbers]{natbib}

%    to redefine the bibliography heading string ("Publications")
% \renewcommand{\refname}{Articles}

%    this produces the bibliography to be numbered decreasingly, fully
%    compatible with natbib and multibib
\newlength{\lmar}
\setlength{\lmar}{\hintscolumnwidth}
\addtolength{\lmar}{\separatorcolumnwidth}
\usepackage[topsep=0pt,leftmargin=\lmar]{etaremune}
\usepackage{etoolbox}
\makeatletter
\AtBeginDocument{%%% natbib redefines the environment there
  \renewenvironment{thebibliography}[1]
  {%
    \bibliographyhead{\refname}%
    \begin{etaremune}
      \@openbib@code%
      \sloppy%
      \clubpenalty4000%
      \widowpenalty4000%
      \sfcode`\.\@m%
      \sfcode`\=1000\relax}%
    {\bibitem@fin\bibpostamble%
      \def\@noitemerr{\latex@warning{Empty `thebibliography' environment}}%
    \end{etaremune}
    \bibcleanup}
}%%% end of \AtBeginDocument
%%% patch \@lbibitem to use only \item (for etaremune)
\patchcmd{\@lbibitem}{\item[\hfil\NAT@anchor{#2}{\NAT@num}]}{\item}{}{}
\makeatother

%%--------------------------------------------------------------------------
%% New macros
\newcommand{\mbs}{Mag\-ne\-to\-brems\-strah\-lung}

\newcommand{\UVval}{Universitat de Val\`{e}ncia}
\newcommand{\UVes}{Universidad de Valencia}
\newcommand{\UVen}{University of Valencia}
\newcommand{\DAAval}{Departament d'Astronomia i Astrof\'{\i}sica}
\newcommand{\DAAes}{Departamentento de Astronom\'{\i}a y Astrof\'{\i}sica}
\newcommand{\DAAen}{Department of Astronomy and Astrophysics}
\newcommand{\IFMes}{Instituto de F\'{\i}sica y Matem\'{a}ticas}
\newcommand{\IFMen}{Institute of Physics and Mathematics}
\newcommand{\UMSNHes}{Universidad Michoacana de San Nicol\'{a}s de Hidalgo}
\newcommand{\UMSNHen}{Michoacan University of Saint Nicholas of Hidalgo}
\newcommand{\UAEMex}{UAEM\'{e}x}
\newcommand{\UAEMes}{Universidad Aut\'{o}noma del Estado de M\'{e}xico}
\newcommand{\UAEMen}{Autonomous University of the State of Mexico}

\newcommand{\prd}{Phys. Rev. D}
\newcommand{\mnras}{Mon. Not. R. Astron. Soc.}

%---------------------------------------------------------------------------
%            content
%---------------------------------------------------------------------------
%

\begin{document}

% ---------------------------   resume   -----------------------------------
\makecvtitle%

\section{Education}

\cventry{2011--}{PhD in Physics}{\UVval}{Valencia, Spain}{}{Supervisors:
  Prof.\ Miguel\ \'{A}ngel Aloy Tor\'{a}s and Dr.\ Petar\ Mimica \newline%
  Thesis: \textit{Radiation Transport in Relativistic Magnetized Fluids ---
    Applications to Relativistic Outflows}}
%
\cventry{2009--2011}{MSc in Physics}{\IFMes, \UMSNHes}{Morelia,
  Michoacan, Mexico}{}{Supervisor: Prof.\ Jos\'{e} Antonio Gonz\'{a}lez
  Cervera \newline%
  Thesis: \textit{Study of TOV stars with the SPH method}}
%
\cventry{2004--2009}{BSc in Physics}{\UAEMes}{Toluca, State of
  Mexico, Mexico}{}{Supervisor: Prof.\ Francisco S. Guzm\'{a}n Murillo
  \newline%
  Thesis: \textit{Numerical solution of null geodesics for the generation
    of gravitational lenses in spherically symmetric space-times}}
% arguments 3 to 6 can be left empty


%%%%%%%%%%%%%%%%%%%%%%%%%%%%%%%%%%%%%%%%%%%%%%%%%%%%%%%%%%
%%%%%%%%%%%%%%%%%%%%%%%%%%%%%%%%%%%%%%%%%%%%%%%%%%%%%%%%%%


\section{Computer skills}

\cvitem{Proficient}{Unix (Linux, macOS), FORTRAN (77, 90, 95, 2003),
  Python, Shell, Mathematica, \LaTeX, gnuplot, grace, OpenMP, GeoGebra,
  Emacs, HDF5, Makefile, Git}
%
\cvitem{Intermediate}{C, C++, R, Julia, Elisp, MPI, SageMath, yEd,
  OpenOffice, Microsoft Office, iWork, DOT, TikZ/PGF}
%
\cvitem{Basic}{HTML, Matlab, Maple, Java, Swift, Perl}


%%%%%%%%%%%%%%%%%%%%%%%%%%%%%  PUBLICATIONS  %%%%%%%%%%%%%%%%%%%%%%%%%%%%%%%


\section{Publications}           % ----- multibib -----
%%\nocitebook{book1,book2}
%%\bibliographystylebook{plain}
%%\bibliographybook{publications}         % 'publications' is the name of a BibTeX file

\nocitejour{Rueda:2017hd,Rueda:2014mn,Guzman:2009bk}
\nociteconf{Rueda:2014sw,Rueda:2013ep,Mimica:2013iu}
\bibliographystylejour{unsrt}
\bibliographystyleconf{unsrt}
\bibliographyjour{MyPapers}
\bibliographyconf{MyPapers}


%%%%%%%%%%%%%%%%%%%%%%%%%%%%%%%%%%%%%%%%%%%%%%%%%%%%%%%%%%%%%%%%%%%%%%%%%%%%


% \section{PhD thesis}
% \cvitem{title}{\textit{}}
% \cvitem{supervisors}{Prof. Miguel A. Aloy\\ Dr. Petar Mimica}
% \cvitem{description}{Short thesis abstract}
%
% \section{MSc thesis}
% \cvitem{title}{\textit{}}
% \cvitem{supervisors}{Prof. Jose A. Gonz\'{a}lez}
% \cvitem{description}{Short thesis abstract}
%
%\section{Bachelors thesis}
%\cvitem{title}{\textit{}}
%\cvitem{supervisor}{Prof. Francisco S. Guzm\'{a}n}
%\cvitem{description}{}


%%%%%%%%%%%%%%%%%%%%%%%%%%%%%%%%%%%%%%%%%%%%%%%%%%%%%%%%%%%%%%%%%%%%%%%%%%%%


\section{Interests}

\cvitem{High energy physics around black holes}{
  \begin{itemize}
  \item Theory and observation of high energy radiation in different
    scenarios where black holes are involved.
    \begin{itemize}
    \item Radiation transport.
    \item Radiation source and source region.
    \item Particles acceleration processes.
    \end{itemize}
  \item Active galactic nuclei.
    \begin{itemize}
    \item Blazars.
      \begin{itemize}
      \item Acceleration processes in the emission region.
      \item Location of the emission region.
      \item The spectral effects due to different constituents of the
        material in the emission region.
      \end{itemize}
    \item Radio galaxies.
    \item Quasars.
    \end{itemize}
  \item Tidal disruption events.
  \item Microquasars.
  \item Gamma-ray bursts.
  \end{itemize}
}
%
\cvitem{Numerical Astrophysics}{
  \begin{itemize}
  \item Numerical solutions to the radiation transport equation with
    astrophysical applications.
  \item Numerical treatment of the microphysics involved in the emission
    of high energy radiation.
  \item Numerical hydrodynamics and magnetohydrodynamics
  \item Performance, stability, convergence and accuracy of numerical
    codes.
  \end{itemize}
}
% \cvitem{hobby 3}{Description}

\clearpage%
\section{Research experience}

\cventry{2011--}{Graduate research assistant}{DAA, UV}{Burjassot,
  Spain}{}{PhD studies
  \begin{itemize}
  \item Parameter study using the code developed by Petar Mimica and Miguel
    A. Aloy for the internal shocks (IS) model.
    \begin{itemize}
    \item Development of software capable of automatizing the launch of
      simulations of ISs
    \item Development of software capable of automatizing the generation of
      plots from the IS code.
    \item Interpretation of lightcurves (LCs) and spectral energy
      distributions (SEDs)
    \item Identification and interpretation of the main physical parameters
      in the shocks.
    \item Identification of the physical parameters in the model which led
      to observational data.
    \end{itemize}
  \item Processing and analysis of data from Fermi 2LAC catalogue.
  \item Identification and interpretation of the main characteristics of
    blazars SEDs (Compton dominance, syncrotron and Compton peaks, spectral
    index).
  \item Calculation of the spectral index from SED data, specifically in
    the \SIrange{0.1}{10}{\giga\electronvolt} band.
  \item Parabolic fitting of SEDs.
  \item Injection of a hybrid thermal-nonthermal distribution of particles
    in the IS model.
  \item Calculation of \mbs\ tables of charged particles of arbitrary
    velocity.
  \item Calculation of the emissivity for isotropic distributions of
    particles using \mbs\ tables.
  \item Implementation of \mbs\ and hybrid distributions to the ISs code.
  \item Contribution to the writing of two manuscript for publication in a
    peer-reviewed journal.
  \end{itemize}
}
%
\cventry{2010--2011}{Graduate research assistant}{IFM, UMSNH}{Morelia,
  Mexico}{}{Master thesis project%
  \begin{itemize}
  \item Writing of a Newtonian smoothed-particle hydrodynamics (SPH) code.
  \item Implementation of a Predictor-Corrector method for the time
    evolution of the hydrodynamic equations.
  \item Implementation of a Newton-Rapson method for the recovery of the
    hydrodynamic primitive variables.
  \item Solving of the Sod shock tube.
  \item Solving of an isothermal collapse.
  \item Writing of a relativistic SPH code.
  \item Solving of the relativistic Sod shock tube.
  \item Writing of the TOV field equations.
  \item Solution of the TOV equations with an RK4 code to generate the
    initial conditions of the SPH code.
  \end{itemize}
}
%
\cventry{2008--2009}{Graduate research assistant}{Faculty of Sciences,
  \UAEMex}{Toluca, Mexico}{}{Bachelor thesis project%
  \begin{itemize}
  \item Writing of the geodesic equation for a spherically symmetric and
    static space-time.
  \item Writing of a fourth order Runge-Kutta solver (RK4)
  \item Testing of the RK4 with ordinary differential equations with well
    known analytic solution
  \item Characterization of the RK4 code
    \begin{itemize}
    \item Convergence
    \item Stability
    \end{itemize}
  \item Application of RK4 to the geodesics equation in a Schwarzschild
    space-time
  \item Implementation of a first order interpolation routine for the
    Christoffel symbols from a numerical metric.
  \item Implementation of the code to the 
  \item Application of RK4 to the geodesics equation in a Boson stars
    (numerical) solution of Einstein's field equations.
  \item Characterization of gravitational lenses around:
    \begin{itemize}
    \item Black holes,
    \item Boson stars.
    \end{itemize}
  \item Interpretation of light trajectories due to curved space-times.
  \item Contributing to the writing of a manuscript for publication in a
    peer-reviewed journal.
  \end{itemize}
}
%
\cventry{2007--2008}{Undergraduate research assistant}{Faculty of Sciences,
  \UAEMex}{Toluca, Mexico}{}{Internship service project\newline%
  Supervisor: Prof.\ Jorge Orozco Velasco.%
  \begin{itemize}
  \item Writing the elliptic equations in finite differences form
  \item Characterization of the typical kinds of boundary conditions:
    \begin{itemize}
    \item Dirichlet
    \item Neumann
    \end{itemize}
  \item Writing of a code which solves the two-dimensional Laplace equation
    in Cartesian coordinates with Dirichlet and Neumann boundary
    conditions.
  \end{itemize}
}
%
\cventry{25 Jun--24 Aug 2007}{Undergraduate research assistant}{Mexican
  Academia of Science}{Morelia, Mexico}{}{National program for temporary
  stays at national research centers for undergraduate science
  students.\newline%
  Supervisor: Prof.\ Francisco S. Guzm\'{a}n Murillo.%
  \begin{itemize}%
  \item Numerical solution of the wave equation with finite differences.
  \item Numerical solution of Burgers' equation with finite differences.
  \item Numerical solution of the general relativistic one-dimensional wave
    equation in the 3+1 formalism with finite differences.
  \end{itemize}
}
%
\cventry{2005--2008}{Undergraduate researcher assistant}{Faculty of
  Sciences, \UAEMex}{Toluca, Mexico}{}{Volunteer work in a faculty research
  project\newline%
  Supervisor: Prof.\ Porfirio D. Rosendo-Francisco
  \begin{itemize}
  \item Exposure of graphite samples to microwaves
    \begin{itemize}
    \item Ultrasonic cleaning of graphite samples.
    \item Systematic exposure graphite samples to microwaves
      ($\SI{2.45}{\giga\hertz}$).
    \item Observation of the superficial effects using a metallographic
      microscope.
    \item Characterization of the structures observed.
      % \item Results presented in a poster at the XLVIII National Physics
      %   Meeting, Guadalajara, M\'{e}xico, 2005.
    \end{itemize}
  \item Exposure of graphite samples to elecric arcs%
    \begin{itemize}
    \item Ultrasonic cleaning of graphite samples.
    \item Characterization of a Tesla coil.
      \begin{itemize}
      \item Input current.
      \item Output flux of electrons.
      \end{itemize}
    \item Controlled handling of a Tesla coil.
    \item Systematic exposure of the surface of graphite samples to a
      perpendicular and tangential electric arc.
    \item Observation of surface effects with a metallographic microscope.
    \item Characterization of the zones around the contact region.
    \item Characterization of the temperature around the contact region.
    \item Characterization of the structures which appeared after the
      exposure.
    \item Analysis of X-rays spectra of the samples.
    \item Identification of induced families of lattice planes.
      % \item Results presented in a poster at the XLIX National Physics Meeting,
      %   San Luis Potosi, M\'{e}xico, 2006.
      % \item Results presented in a poster at the L National Physics
      %   Meeting, Boca del R\'{\i}o, M\'{e}xico, 2007.
    \end{itemize}
  \end{itemize}
}
% \cventry{year--year}{Job title}{Employer}{City}{}{Description line 1\newline{}Description line 2
% \item Achievement 2, with sub-achievements:
%   \begin{itemize}%
%   \item Sub-achievement (a);
%   \item Sub-achievement (b), with sub-sub-achievements (don't do this!);
%     \begin{itemize}
%     \item Sub-sub-achievement i;
%     \item Sub-sub-achievement ii;
%     \item Sub-sub-achievement iii;
%     \end{itemize}
%   \item Sub-achievement (c);
%   \end{itemize}
% \item Achievement 3.
% }
%   \subsection{Miscellaneous}
%   \cventry{year--year}{Job title}{Employer}{City}{}{Description}


%%%%%%%%%%%%%%%%%%%%%%%%%%%%%%%%%%%%%%%%%%%%%%%%%%%%%%%%%%%%%%%%%%%%%%%%%%%%


% \subsection{Teaching}
% \cventry{year--year}{}{}{}{}{}


%%%%%%%%%%%%%%%%%%%%%%%%%%%%%%%%%%%%%%%%%%%%%%%%%%%%%%%%%%%%%%%%%%%%%%%%%%%%


\section{Meetings and conferences}

\subsection{Oral presentations}

\cventry{2014}{Rueda-Becerril, J.M.\textnormal{; Mimica, P.; Aloy,
    M.A.}}{Numerical simulations of the internal shock model in
  magnetized relativistic jets of blazars}{IVICFA's Fridays: Computation
  in Physics}{Paterna, Spain, 17 October}{}
%
\cventry{2014}{Rueda-Becerril, J.M.\textnormal{; Mimica, P.; Aloy,
    M.A.}}{Influence of the magnetic field on the spectral properties of
  blazars in the internal shocks scenario}{Extreme-Astrophysics in an
  Ever-Changing Universe: Time-Domain Astronomy in the 21st
  Century}{Ier\'{a}petra, Greece, 16--20 June}{}
%
\cventry{2013}{Rueda-Becerril, J.M.\textnormal{; Mimica, P.; Aloy,
    M.A.}}{Numerical study of broadband spectra caused by internal shocks
  in magnetized relativistic jets}{XXXIV Biennial meeting of the Royal
  Spanish Society of Physics}{Valencia, Spain, 15--19 July}{}
%
\cventry{2009}{Rueda-Becerril, J.M}{\textquestiondown{}Dec\'{\i}a Einstein
  la verdad?}{weekly colloquium of Physics students \emph{Caf\'{e}
    Ciencias}}{Toluca, Mexico, 11 March}{}

\subsection{Poster presentations}

\cventry{2014}{Rueda-Becerril, J.M.\textnormal{; Mimica, P.; Aloy,
    M.A.}}{Numerical simulations of the internal shock model in
  magnetized relativistic jets of blazars}{Swift: 10 years of
  Discovery}{Rome, Italy, 2--5 December}{}
%
\cventry{2013}{Rueda-Becerril, J.M.\textnormal{; Mimica, P.; Aloy,
    M.A.}}{Numerical study of broadband spectra caused by internal shocks
  in magnetized relativistic jets of blazars}{The Innermost Regions of
  Relativistic Jets and Their Magnetic Fields}{Granada, Spain, 10--14 June}{}
%
\cventry{2007}{Rueda-Becerril, J.M.\textnormal{; Leyte Gonz\'{a}lez, R.;
    Garc\'{\i}a Santiba\~{n}ez, F.; Rosendo-Francisco, P.}}{Analysis of
  the superficial structure of graphite samples submitted to an electric
  arc}{L National Physics Meeting}{Boca del R\'{\i}o, Mexico, 29
  October--2 November}{}
%
\cventry{2006}{Rueda-Becerril, J.M.\textnormal{; Leyte Gonz\'{a}lez, R.;
    Garc\'{\i}a Molina, N.; Rosendo-Francisco, P.}}{Modifications on the
  superficial structure of graphite samples}{XLIX National Physics
  Meeting}{San Luis Potos\'{\i}, Mexico, 16--19 October}{}
%
\cventry{2005}{Rueda-Becerril, J.M.\textnormal{; G\'{o}mez D\'{\i}az, A.;
    Rosendo-Francisco, P.}}{Studies of microwave effects of graphite
  samples}{XLVIII National Physics Meeting}{Guadalajara, Mexico, 17--21
  October}{}

\subsection{Attendance only}

\cvitem{2016}{CoCoNuT Meeting 2016, Burjassot, Spain, 14--16 December}
%
\cvitem{2008}{LI National Physics Meeting, Zacatecas, Mexico, 20--24
  October}

\subsection{Organization}

\cvitem{2012}{Contribution to the organization of the X Scientific Meeting
  of the Spanish Astronomical Society, Valencia, Spain, 14--16 December}


\section{Professional development}

\cventry{7--16 Feb 2017}{Data Analysis and Machine Learning with
  Python}{UV}{Burjassot}{Spain}{No.\ of hours: 8}
%
\cventry{23--16 May 2014}{The Universe in the light of PLANCK and
  BICEP2}{UV}{Burjassot}{Spain}{No.\ of credits: 2}
%
\cventry{23--27 Sep 2013}{Dark Matter}{UV}{Burjassot}{Spain}{No.\ of
  credits: 2}
%
\cventry{23 Apr--8 May 2013}{International Cag\`{e}se School on Cosmic
  Accelerators}{Institut d'\'{E}tudes Scientifques de
  Carg\`{e}se}{Carg\`{e}se}{France}{}
%
\cventry{9--12 Apr 2012}{Introduction to C++
  Programming}{UV}{Burjassot}{Spain}{No.\ of credits: 6}
%
\cventry{27 Mar--4 Apr 2012}{Numerical Relativistic
  Astrophysics}{UV}{Burjassot}{Spain}{No.\ of hours: 9}
%
\cventry{5--9 March 2012}{Fortran for Scientific Computing}{High
  Performance Computing Center Sttutgart}{Stuttgart}{Germany}{No.\ of
  hours: 33}
%
\cventry{Jun 2006}{Advanced Summer School}{CINVESTAV}{Ciudad de
  M\'{e}xico}{Mexico}{}
%
\cventry{Aug 2006}{Advanced Summer School}{Instituto de F\'{\i}sica of the
  Universidad de Guanajuato}{Le\'{o}n}{Mexico}{}

\clearpage%
\section{Awards and Scholarships}

\cvitem{2014--2016}{\textbf{Fellowship} from the Mexican Federal Government
  to study abroad awarded by the National Council of Science and Technology
  (CONACyT).}
% 
\cvitem{2011--2014}{ \textbf{Fellowship} ``Santiago Grisol\'{\i}a'' awarded
  by the Council of Education, Research, Culture and Sport of the Valencian
  Comunity.}
%
\cvitem{2009--2011}{\textbf{Fellowship} for academic training for MSc
  studies granted by the Mexican Council of Science and Technology
  (CONACyT).}
%
\cvitem{2009}{\textbf{Award} ``Lic. Juan Josafat Pichardo Cruz'', granted
  by the \UAEMex, for finishing the BSc thesis and graduating within a year
  after completing the undergraduate credits.}
%
\cvitem{25 Jun--24 Aug 2007}{\textbf{Fellowship} for a temporary stay in a
  national research center under the XVII summer of scientific
  investigation program awarded by the Mexican Academia of Science.}


\section{Other activities}
\cvitem{Aug 2007--May 2009}{Physics students representative at the
  Governing Council of the Faculty of Sciences of the \UAEMex}{}{}{}{}


\section{Languages}

% \cvitemwithcomment{Language 3}{Skill level}{Comment}
\cvitemwithcomment{Spanish}{Mother tongue}{}
\cvitemwithcomment{English}{Proficient}{}
\cvitemwithcomment{Catalan}{Basic}{}%IELTS and TOEFL certified.}
\cvitemwithcomment{French}{Basic}{}%IELTS and TOEFL certified.}
\cvitemwithcomment{German}{Basic}{}%IELTS and TOEFL certified.}


\section{References}

\cvlistitem{Prof.\ Miguel \'{A}ngel Aloy Tor\'{a}s\ \ $\cdotp$\ \ +34 96
  354 3080\ \ $\cdotp$\ \
  \href{mailto:Miguel.A.Aloy@uv.es}{Miguel.A.Aloy@uv.es}\ \ $\cdotp$\ \ UV}
%
\cvlistitem{Dr.\ Petar Mimica\ \ $\cdotp$\ \ +34 96 354 3080\ \ $\cdotp$\ \
  \href{mailto:Petar.Mimica@uv.es}{Petar.Mimica@uv.es}\ \ $\cdotp$\ \ UV} 
%
\cvlistitem{Prof.\ Francisco Siddhartha Guzm\'{a}n Murillo\ \ $\cdotp$\ \
  +52 443 322 3500 ext 1264\ \ $\cdotp$\ \
  \href{mailto:guzman@ifm.umich.mx}{guzman@ifm.umich.mx}\ \ $\cdotp$\ \
  IFM, UMSNH}
%
\cvlistitem{Prof.\ Jos\'{e} Antonio Gonz\'{a}lez Cervera\ \ $\cdotp$\ \
  +52 443 322 3500 ext 1263\ \ $\cdotp$\ \
  \href{mailto:gonzalez@ifm.umich.mx}{gonzalez@ifm.umich.mx}\ \ $\cdotp$\ \
  IFM, UMSNH}
%
\cvlistitem{Prof.\ Jorge Orozco Velasco\ \ $\cdot$\ \
  \href{mailto:jov@uaemex.mx}{jov@uaemex.mx}\ \ $\cdot$\ \ \UAEMex}

% Publications from a BibTeX file without multibib
%  for numerical labels: \renewcommand{\bibliographyitemlabel}{\@biblabel{\arabic{enumiv}}}% CONSIDER MERGING WITH PREAMBLE PART
%  to redefine the heading string ("Publications"): \renewcommand{\refname}{Articles}

%\nocite{*}
%\bibliographystyle{plain}
%\bibliography{Biblio}                        % 'publications' is the name of a BibTeX file


%\clearpage
% ---------------------       letter       ---------------------------------
%       recipient data
%\recipient{Company Recruitment team}{Company, Inc.\\123 somestreet\\some city}
%\date{January 01, 1984}
%\opening{Dear Sir or Madam,}
%\closing{Yours faithfully,}
%\enclosure[Attached]{curriculum vit\ae{}}          % use an optional argument to use a string other than "Enclosure", or redefine \enclname
%De aqui
%\makelettertitle
%
%Lorem ipsum dolor sit amet, consectetur adipiscing elit. Duis ullamcorper neque sit amet lectus facilisis sed luctus nisl iaculis. Vivamus at neque arcu, sed tempor quam. Curabitur pharetra tincidunt tincidunt. Morbi volutpat feugiat mauris, quis tempor neque vehicula volutpat. Duis tristique justo vel massa fermentum accumsan. Mauris ante elit, feugiat vestibulum tempor eget, eleifend ac ipsum. Donec scelerisque lobortis ipsum eu vestibulum. Pellentesque vel massa at felis accumsan rhoncus.
%
%Duis sit amet magna ante, at sodales diam. Aenean consectetur porta risus et sagittis. Ut interdum, enim varius pellentesque tincidunt, magna libero sodales tortor, ut fermentum nunc metus a ante. Vivamus odio leo, tincidunt eu luctus ut, sollicitudin sit amet metus. Nunc sed orci lectus. Ut sodales magna sed velit volutpat sit amet pulvinar diam venenatis.
%
%
%


\end{document}


%%% Local Variables:
%%% mode: latex
%%% TeX-master: t
%%% End:
